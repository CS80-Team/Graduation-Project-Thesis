\documentclass{book}
\usepackage[utf8]{inputenc}
\usepackage{graphicx}
\usepackage{caption}
\usepackage{subcaption}
\usepackage{tikz}
\usepackage{amsmath}
\usepackage{longtable}
\usepackage{titling}

\title{
Continuous File Synchronization System 
}


\begin{document}
	
	\maketitle
	
	\chapter{Introduction}
		\section{Motivation}
      In today's digital age, the need for seamless, real-time file synchronization across devices has become essential. With the growth of remote work, distributed teams, and multi-device use, individuals and organizations require reliable solutions that ensure data is accessible and up-to-date without manual intervention. While cloud storage providers offer some solutions, they often come with privacy concerns, limitations on customization, and dependency on third-party servers.

      To address these challenges, our graduation project focuses on developing a robust file synchronization system inspired by platforms like Syncthing, aiming for a decentralized, peer-to-peer approach that priotrizes user privacy, security, and ease of use. our intention is to eliminate reliance on centralized servers, providing users with greater control over their data while ensuring conflict-free file synchronization across devices. Through this system, we seek to contribute to the field of distributed computing by offering an alternative that enhances data accessibility and resilience

		\section{Problem Statement}
		
		\section{Goals}
		
		\section{Related Work}
			\subsection{Currently Available Solutions}
			
			\subsection{Features Matrix}
			
		\section{Software Development Methodology Used}
	
	\chapter{Requirements Analysis}
		\section{Surveys}
		
		\section{Results of (Part 2.1) represented as charts}
		
		\section{Functional Requirements}
		
		\section{Non-functional Requirements}
		
		\section{Use Case Diagrams}
	
	\chapter{Design}
		\section{Sequence Diagrams}
		
		\section{ERD}
		
		\section{Data Flow Diagrams (DFD-level 0 and DFD-level 1)}
		
		\section{Algorithms}
		
		\section{UML Class Diagrams}
	
	\chapter{Implementation Aspects}
		\section{Overall System Architecture (Blocks diagram)}
		
		\section{Tools, Technologies and/or Programming Languages}
		
		\section{Prototype}
	
\end{document}
